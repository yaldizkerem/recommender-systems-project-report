Recommender systems are used to estimate rating or preference that a user will assign to an item. Almost every major tech comopany has applied them. Amazon uses it to suggest products, Facebook uses it to recommend people to follow, Netflix uses it to suggest movies and series to watch. Netflix even gave a million dollars to anyone who could increase its system by 10\%.

\begin{figure}[h!]
   \centering
   \includegraphics[width=0.8\linewidth]{images/recommender-systems.png}
   \caption{Recommender Systems~\cite{recommendersystem}}
   \label{fig:flowchart}
\end{figure}

Recommender systems have changed the way web sites communicate with their users. Rather than providing content, based on users' search, provides a richer experience by indentifying recommendations for every user based on past searches, purchases and votes.

Nowadays nearly all major product selling web sites have implemented recommender systems in one way or another. Most of them recommends products using visited products such as "who bought this product also bought these products!". Some of them recommends products based on users' activity such as keeping track of visited products. A remarkable amount of web sites computes similarity between users and uses this information for recommendation. Some of these web sites uses some or all of these methods to promote purchase.~\cite{jannach2010recommender}

There are majorly six types of recommender systems: collaborative, content-based, demographic based, utility based, Knowledge based and Hybrid recommender system. One of the most popular and promising type is collobrative

There are several types of recommender systems. One of the most promising is the collobrative filtering aproach. It provides recommendations using only user-item interactions. It's divided into two types; item based or user based.~\cite{xie2015link} Collobrative filtering aproaches greatly suffers from data sparcity problem.~\cite{chen2005link}

Collaborative filtering is a great recommendatiton aproach  that uses past user-item interactions for recommendation. By mapping interactions to a bipartite user-item interaction graph, a recommendation problem is converted to a link prediction problem.~\cite{xie2015link}

In this paper, I implemented \ac{CORLP} method using MovieLens dataset.